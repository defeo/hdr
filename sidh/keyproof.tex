We recall the definition of session-key security in the
authenticated-links adversarial model of Canetti and
Krawczyk~\cite{canetti}. We consider a finite set of \emph{parties}
$P_1, P_2, \ldots, P_n$ modeled by probabilistic Turing machines.  The
adversary $\mathcal{I}$, also modeled by a probabilistic Turing
machine, controls all communication, with the exception that the
adversary cannot inject or modify messages (except for messages from
corrupted parties or sessions), and any message may be delivered at
most once. Parties give outgoing messages to the adversary, who has
control over their delivery via the \textsf{Send} query. Parties are
activated by \textsf{Send} queries, so the adversary has control over
the creation of protocol sessions, which take place within each party.
Two sessions $s$ and $s'$ are \emph{matching} if the outgoing messages
of one are the incoming messages of the other, and vice versa.

We allow the adversary black-box access to the queries
$\mathsf{SessionStateReveal}$, $\mathsf{SessionKeyReveal}$, and
$\mathsf{Corrupt}$. The $\mathsf{SessionStateReveal}(\mathfrak{s})$
query allows the adversary to obtain the contents of the session
state, including any secret information. The query is noted and
$\mathfrak{s}$ produces no further output.  The
$\mathsf{SessionKeyReveal(\mathfrak{s})}$ query enables the adversary
to obtain the session key for the specified session $\mathfrak{s}$, so
long as $\mathfrak{s}$ holds a session key. The
$\mathsf{Corrupt(P_i)}$ query allows the adversary to take over the
party $P_i$, i.e.,\ the adversary has access to all information in
$P_i$'s memory, including long-lived keys and any session-specific
information still stored. A corrupted party produces no further
output. We say a session $\mathfrak{s}$ with owner $P_i$ is
\emph{locally exposed} if the adversary has issued
$\mathsf{SessionKeyReveal(\mathfrak{s})}$,
$\mathsf{SessionStateReveal(\mathfrak{s})}$, or
$\mathsf{Corrupt(P_i)}$ before $\mathfrak{s}$ is expired. We say
$\mathfrak{s}$ is \emph{exposed} if $\mathfrak{s}$ or its matching
session have been locally exposed, and otherwise we say $\mathfrak{s}$
is \emph{fresh}.

We allow the adversary $\mathcal{I}$ a single
$\mathsf{Test(\mathfrak{s})}$ query, which can be issued at any stage
to a completed, fresh, unexpired session $\mathfrak{s}$. A bit $b$ is
then picked at random. If $b=0$, the test oracle reveals the session
key, and if $b=1$, it generates a random value in the key
space. $\mathcal{I}$ can then continue to issue queries as desired,
with the exception that it cannot expose the test session. At any
point, the adversary can try to guess $b$. Let
$\operatorname{GoodGuess}^{\mathcal{I}}(k)$ be the event that
$\mathcal{I}$ correctly guesses $b$, and define \[
\operatorname{Advantage}^{\mathcal{I}}(k) = \max \left\{0, \left \vert
\operatorname{Pr}[\operatorname{GoodGuess}^{\mathcal{I}}(k)] - \frac 1
2 \right \vert \right\},\] where $k$ is a security parameter.

The definition of security is as follows:

\begin{definition}\label{def:kep}
A key exchange protocol $\Pi$ in security parameter $k$ is said to be
\emph{session-key secure} in the authenticated-links adversarial model
of Canetti and Krawczyk if for any polynomial-time adversary $\mathcal{I}$,
\begin{enumerate}
\item If two uncorrupted parties have completed matching sessions,
  these sessions produce the same key as output;
\item $\operatorname{Advantage}^{\mathcal{I}}(k)$ is negligible.
\end{enumerate}
\end{definition}

\begin{algorithm}[t]
\caption{SSDDH distinguisher}
\label{alg:distinguisher}
\begin{algorithmic}[1]
\REQUIRE $E_A, E_B, \phi_A(P_B), \phi_A(Q_B), \phi_B(P_A),
\phi_B(Q_A), E$
\STATE $r \stackrel{R}{\leftarrow} \{1,\ldots,k\}$, where $k$ is an
upper bound on the number of sessions activated by $\mathcal{I}$ in any
interaction.
\STATE Invoke $\mathcal{I}$ and simulate the protocol to $\mathcal{I}$, except for the
$r$-th activated protocol session.
\STATE For the $r$-th session, let Alice send $A$, $i$, $E_A$, $\phi_A(P_B)$,
$\phi_A(Q_B)$ to Bob, and let Bob send $B$, $i$, $E_B$, $\phi_B(P_A)$,
$\phi_B(Q_A)$ to Alice, where $i$ is the session identifier.
\IF{the $r$-th session is chosen by $\mathcal{I}$ as the test session}
\STATE Provide $\mathcal{I}$ as the answer to the test query.
\STATE $d \leftarrow \mathcal{I}$'s output.
\ELSE
\STATE $d \stackrel{R}{\leftarrow}\{0,1\}$.
\ENDIF
\ENSURE $d$

\end{algorithmic}
\end{algorithm}

\begin{proof}[Proof of Theorem~\ref{thm:kep-proof}]
We adapt the proof given by Canetti and Krawczyk~\cite[\S
  5.1]{canetti} for two-party Diffie-Hellman over $\ZZ_q^*$. A similar
strategy was used by Stolbunov~\cite{stolbunov-red} in the case of
ordinary elliptic curves.

It has been shown in Section~\ref{sec:kep} that two uncorrupted
parties in matching sessions output the same session key, and thus the
first part of Definition~\ref{def:kep} is satisfied. To show that the
second part of the definition is satisfied, assume that there is a
polynomial-time adversary $\mathcal{I}$ with a non-negligible advantage
$\varepsilon$. We claim that Algorithm~\ref{alg:distinguisher} forms a
polynomial-time distinguisher for SSDDH having non-negligible
advantage.

To prove the claim, we must show that
Algorithm~\ref{alg:distinguisher} has non-negligible advantage (it is
clear that it runs in polynomial time). We consider separately the
cases where the $r$-th session is (respectively, is not) chosen by
$\mathcal{I}$ as the test session. If the $r$-th session is not the
test session, then Algorithm~\ref{alg:distinguisher} outputs a random
bit, and thus its advantage in solving the SSDDH is $0$. If the
$r$-th session is the test session, then $\mathcal{I}$ will succeed
with advantage $\varepsilon$, since the simulated protocol provided to
$\mathcal{I}$ is indistinguishable from the real protocol. Since the
latter case occurs with probability $1/k$, the overall advantage of
the SSDDH distinguisher is $\varepsilon/k$, which is non-negligible.
\end{proof}


\documentclass{book}

\usepackage[a4paper]{geometry}
\usepackage[english]{babel}
\usepackage{amsmath,amsfonts,amssymb}
\usepackage{hyperref}
\hypersetup{
  unicode=true,
  colorlinks=true,
  citecolor=blue!70!black,
  filecolor=black,
  linkcolor=red!70!black,
  urlcolor=blue,
  pdfstartview={FitH},
  pdfauthor={Luca De Feo},
  pdfsubject={Mathematics},
  pdfkeywords={Cryptography, Number theory, Computer algebra, Elliptic curves, Isogenies, Finite fields},
}

\title{Dossier de candidature HDR}
\author{Luca De Feo\\
  \\
  Équipe CRYPTO\\
  Laboratoire de Mathématiques de Versailles\\
  Université Paris Saclay -- UVSQ}

\date{}

\begin{document}
\frontmatter
\maketitle
\pagestyle{empty}

\chapter*{Curriculum vitæ}

\begin{large}
  Luca De Feo\\
  \\
  born May 26, 1983. Italian.\\
  Maître de Conférences (associate professor) in Computer Science.
\end{large}

\bigskip

\begin{tabular}{l l}
  \begin{minipage}[t]{0.5\linewidth}
    \textbf{Address:}\\
    Université Paris Saclay -- UVSQ\\
    Équipe CRYPTO -- Laboratoire LMV\\
    bureau 309 B\\
    45 Avenue des États-Unis\\
    78035 Versailles Cedex, France\\
  \end{minipage}
  &
    \begin{minipage}[t]{0.5\linewidth}
      \begin{description}
        \setlength{\itemsep}{-0.5ex}
      \item[tel:] +33 1 39 25 40 35
      \item[email:] luca.de-feo@uvsq.fr
      \item[ORCiD:] 0000-0002-9321-0773
      \item[Twitter:] \href{https://twitter.com/@luca\_defeo}{@luca\_defeo}
      \item[GitHub:] \href{https://github.com/defeo}{@defeo}
      \item[Keybase:] \href{https://keybase.io/defeo}{defeo}
      \item[www:] \url{https://defeo.lu/}
      \end{description}
    \end{minipage}
\end{tabular}

\section*{Professional experience}

\subsection*{Employment}

\begin{tabular}{l p{0.8\linewidth}}
  Sep 2016 -- Aug 2018 & Inria Saclay, \textbf{Chercheur délégué
                         (Invited researcher)}.\\
  Dec 2015 -- present & Laboratoire de
                        Mathématiques de Versailles, UVSQ \textbf{Maître de
                        Conférences (Assistant Professor)}.\\
  Sep 2011 -- Nov 2015 &
                         Laboratoire PRiSM, UVSQ \textbf{Maître de
                         Conférences (Assistant Professor)}.\\
  Jul 2011 -- Aug 2011 &
                         Combinatorics \& Optimization, University of Waterloo
                         \textbf{Postdoctoral researcher}. Supervisor: David Jao.\\
  Dec 2010 -- May 2011 & IRMAR,
                         Université de Rennes 1 \textbf{Postdoctoral researcher}.
                         Supervisor: Reynald Lercier.
\end{tabular}

\subsection*{Education}

\begin{tabular}{l p{0.9\linewidth}}
  2007--2010 & École Polytechnique, \textbf{PhD in Computer
               Science}, \emph{Fast Algorithms for Towers of Finite Fields and
               Isogenies}. Supervisors: François Morain and Éric Schost.\\
  2004--2007 & ENS Ulm (Paris), \textbf{Diplôme de
               l'École Normale Supérieure}, \emph{major Computer Science}, \emph{minor
               Mathematics}.\\
  2005--2007 & ENS Ulm (Paris), \textbf{Master
               Parisien de Recherche en Informatique}, \emph{grade: Bien}. \\
  2004--2005
             & ENS Ulm (Paris), \textbf{Bachelor (licence) of Mathematics}.\\
  2001--2004 & Università di Pisa, \textbf{Bachelor (laurea
               triennale) of Computer Science}, \emph{110/110 cum laude}.
\end{tabular}


\section*{Research activity}

\textbf{Keywords:} Algorithmic Number Theory, Computer Algebra,
Cryptology, Mathematical Software.

\subsection*{About}

I research and develop algorithms for \textbf{finite fields} and
\textbf{algebraic curves}. I tend to put the stress on
\textbf{efficient algorithms}, both in the theoretical analysis and
the software implementation.

I am especially interested in applications to cryptology, in
particular \textbf{elliptic curve cryptography} and \textbf{isogeny
  based cryptography}.  My research encompasses the whole production
chain of cryptograhic systems: from the initial theoretical
conception, to the software specification, to the security of the
final product.  For example, I have both contributed to design new
cryptosystems based on isogenies, and to produce a candidate
specification for an encryption scheme submitted to the National
Institute of of Standards and Technology (NIST).

I am very much interested in \textbf{software}, and I am a strong
believer in open source. My software production, both related and
unrelated to research, is plethoric: 100 repositories and counting on
my personal GitHub page (\url{https://github.com/defeo/}), although
none of these could be qualified as ``popular''.

More to the point, I make a heavy use of \textbf{computer algebra
  systems}, both in research and teaching, and I am a longtime
contributor to SageMath (\url{https://www.sagemath.org}), one of the
most popular open source software for doing mathematics. Part of my
contributions can be seen at
\url{https://trac.sagemath.org/search?q=defeo}. On GitHub, I am member
of the SageMath, OpenMath and OpenDreamKit organizations, among
others.

My involvement in the SageMath community has led me to take part to
the H2020 Research Infrastructures project
\href{https://opendreamkit.org/}{OpenDreamKit}, whose goal is to
enhance the ecosystem of computational mathematics.  I have done in
OpenDreamKit some of my most important, and most invisible, work:
modernizing, rethinking, and consolidating the foundations of the most
used open source tools in Mathematics (SageMath, GAP, PARI/GP,
Singular, to name a few).


\subsection*{PhD supervision}

\paragraph{Cyril Hugounenq}
\begin{quote}
  \begin{description}
    \setlength{\itemsep}{-0.5ex}
  \item[Title:] Volcans et calcul d'isogénies.
  \item[Doctoral school:] Computer science.
  \item[Date started:] October 2013.
  \item[Date defended:] September 2017.
  \item[Funding:] Digiteo doctoral grant, \emph{Projet ARGC:
      Arithmetique Rapide pour la Géométrie et la Cryptologie}.
  \item[Supervisors:] Luca De Feo 80\%, Louis Goubin 10\%, François
    Morain 10\%.
  \item[Joint publications:]\
    \begin{itemize}
    \item Luca De~Feo, Cyril Hugounenq, J{\'e}r{\^o}me Pl{\^u}t, and {\'E}ric Schost.
      \newblock Explicit isogenies in quadratic time in any characteristic.
      \newblock {\em LMS Journal of Computation and Mathematics}, 19(A):267--282,
      2016.
    \end{itemize}
  \item[Current employment:] Postdoc, Université Joseph Fourier,
    Grenoble 1.
  \end{description}
\end{quote}

\paragraph{Robin Larrieu}
\begin{quote}
  \begin{description}
    \setlength{\itemsep}{-0.5ex}
  \item[Title:] Fast Finite Field Arithmetic
  \item[Doctoral school:] Computer science.
  \item[Date started:] October 2016.
  \item[Planned defense:] September 2019.
  \item[Funding:] École Polytechnique.
  \item[Supervisors:] Luca De Feo 30\%, Joris van der Hoeven 70\%.
  \end{description}
\end{quote}

\paragraph{Édouard Rousseau}
\begin{quote}
  \begin{description}
    \setlength{\itemsep}{-0.5ex}
  \item[Title:] Arithmétique efficace pour la cryptographie et la
    cryptanalyse
  \item[Doctoral school:] Mathematics.
  \item[Date started:] October 2017.
  \item[Planned defense:] September 2020.
  \item[Funding:] DIM Math-Innov, FMJH.
  \item[Supervisors:] Luca De Feo 50\%, Hugues Randriambololona 50\%.
  \item[Joint publications:]\
    \begin{itemize}
    \item Lattices of compatibly embedded finite fields. {\em Software
        presentation}. ISSAC 2018, New York.
    \end{itemize}
  \end{description}
\end{quote}


\subsection*{Master's thesis supervision}

\paragraph{Sébastien Besnier}
\begin{quote}
  \begin{description}
    \setlength{\itemsep}{-0.5ex}
  \item[Title:] Participation au développement d'un framework pour les courbes
         elliptiques en Sage.
  \item[Degree:] Mathematics.
  \item[\emph{Alma mater}] Université de Versailles--Saint-Quentin-en-Yvelines.
  \item[Year:] 2014.
  \item[Current employement] High school teacher (professeur agrégé).
  \end{description}
\end{quote}

\paragraph{Ludovic Brieulle}
\begin{quote}
  \begin{description}
    \setlength{\itemsep}{-0.5ex}
  \item[Title:] Calcul d'isomorphismes de corps finis.
  \item[Degree:] Mathematics.
  \item[\emph{Alma mater}] Université de Versailles--Saint-Quentin-en-Yvelines.
  \item[Year:] 2014.
  \item[Joint publications:]\
    \begin{itemize}
    \item
      Ludovic Brieulle, Luca De~Feo, Javad Doliskani, Jean-Pierre Flori, and {\'E}ric
      Schost.
      \newblock Computing isomorphisms and embeddings of finite fields.
      \newblock {\em Mathematics of Computation}, 2018.
    \end{itemize}
  \end{description}
\end{quote}

\paragraph{Jean Kieffer}
\begin{quote}
  \begin{description}
    \setlength{\itemsep}{-0.5ex}
  \item[Title:] Protocoles d'échange de clefs à base d'isogénies.
  \item[Degree:] Mathematics.
  \item[\emph{Alma mater}] École Normale Supérieure and Université
    Pierre et Marie Curie.
  \item[Year:] 2017.
  \item[Joint publications:]\
    \begin{itemize}
    \item Luca De~Feo, Jean Kieffer, and Benjamin Smith.
      \newblock Towards practical key exchange from ordinary isogeny graphs.
      \newblock In {\em AsiaCrypt 2018}, 2018.
    \end{itemize}
  \item[Current employement] PhD student, Inria Bordeaux--Sud Ouest.
  \end{description}
\end{quote}

\paragraph{Mattia Veroni}
\begin{quote}
  \begin{description}
    \setlength{\itemsep}{-0.5ex}
  \item[Title:] Comptage de points de courbes elliptiques en moyenne
    caractéristique.
  \item[Degree:] Mathematics.
  \item[\emph{Alma mater}] Università di Trento, Italy.
  \item[Year:] 2018.
  \item[Current employement] PhD student, Norwegian University of
    Science and Technology, Trondheim.
  \end{description}
\end{quote}


\subsection*{Research grants}

\paragraph{Digiteo doctoral grant}
\begin{quote}
  \begin{description}
    \setlength{\itemsep}{-0.5ex}
  \item[Project:] ARGC: Arithmetique Rapide pour la Géométrie et la
    Cryptologie.
  \item[Duration:] October 2013 -- September 2016.
  \item[Description:] Three year PhD funding. Hired Cyril Hugounenq.
  \item[Participants:] UVSQ, Inria Saclay.
  \item[Role:] Project leader.
  \end{description}
\end{quote}

\paragraph{Digiteo post-doctoral grant}
\begin{quote}
  \begin{description}
    \setlength{\itemsep}{-0.5ex}
  \item[Project:] IdealCodes: Coppersmith's Method for Coding Theory
    and Cryptography.
  \item[Duration:] September 2014 -- August 2016.
  \item[WWW:] \url{https://idealcodes.github.io/}.
  \item[Description:] Two year postdoc funding. Hired Johan Rosenkilde
    (2014--2015) and Virgile Doucet (2015--2016).
  \item[Participants:] Inria Saclay, UVSQ.
  \item[Role:] Partner.
  \end{description}
\end{quote}

\paragraph{Horizon 2020 EU Research Infrastructures}
\begin{quote}
  \begin{description}
    \setlength{\itemsep}{-0.5ex}
  \item[Project:] OpenDreamKit.
  \item[Duration:] September 2015 -- August 2019.
  \item[WWW:] \url{https://opendreamkit.org/}.
  \item[Description:] OpenDreamKit is a project that brings together a
    range of projects and associate software to create and strengthen
    virtual research environments. The most widely used research
    environment is the Jupyter Notebook from which computational
    research and data processing can be directed. The OpenDreamKit
    project provides interfaces to well established research codes and
    tools so that they can be used seamlessly and combined from within
    a Jupyter Notebook.

    OpenDreamKit also supports open source research codes directly by
    investing into structural improvements and new features to not
    only connect all of these tools but also enrich them, and make
    them more sustainable.

    More concretely, the tools brought together in the OpenDreamKit
    project include mathematical software packages such as SageMath,
    GAP, PARI, Singular, but also simulation tools from materials
    science such as OOMMF.

    Luca De Feo is the only contributor to OpenDreamKit at UVSQ.
  \item[Participants:] Université Paris-Sud, Université de Bordeaux,
    Université Grenoble Alpes, University of Kaiserslautern,
    University of Oxford, University of Silesia, University of St
    Andrews, UVSQ, University of Warwick, Logilab, Simula Research
    Laboratory, Universiteit Gent, European XFEL, FAU Universität
    Erlangen-Nürnberg, University of Leeds, Jacobs University Bremen,
    University of Sheffield, University of Southampton, Universität
    Zurich.
  \item[Role:] Site leader for UVSQ. Work-package leader for WP3:
    Component Architecture. Enrolled for 14PM over the four years.
  \end{description}
\end{quote}

\paragraph{Programme Investissements d'Avenir}
\begin{quote}
  \begin{description}
    \setlength{\itemsep}{-0.5ex}
  \item[Project:] RISQ: Regroupement de l’Industrie française pour la
    Sécurité Post-quantique.
  \item[Duration:] September 2015 -- August 2019.
  \item[WWW:] \url{https://risq.fr/}.
  \item[Description:] The goal of the RISQ project is to boost the
    assets of France in the race to secure the communications of the
    post-quantum world. 
    
    Inside RISQ, the UVSQ team coordinates the submission of
    post-quantum encryption and signature algorithms to the NIST
    competition. It will also contribute to the implementation of the
    algorithms and to the integration in a software suite.
  \item[Participants:] Secure-IC, INRIA Paris, CNRS, Université Pierre
    et Marie Curie, THALES, CEA, INRIA Rhône Alpes, ENS de Lyon,
    Université Claude Bernard Lyon I, UVSQ, CryptoExperts, CS, IRISA
  \item[Role:] Participant. Submitter of the SIKE proposal to the NIST
    competition.
  \end{description}
\end{quote}

\subsection*{Awards}

\begin{tabular}{l p{0.9\linewidth}}
  2017 & \textbf{ISSAC 2017 Best Poster Award} for
         \emph{Isogeny-based cryptography in Nemo: A case study}.\\
  2015 & \textbf{ISSAC 2015 Best Poster Award} for \emph{Deterministic Root
         Finding in Finite Fields}.\\
  2009 & \textbf{SIGSAM ISSAC 2009 Distinguished Student Author Award}
         for \emph{Fast arithmetics in Artin-Schreier towers over Finite Fields}.
\end{tabular}

\section*{Teaching}

\subsubsection*{Overview}

\begin{tabular}{l p{0.9\linewidth}}
  2013--present & UVSQ, \textbf{Lecturer} in
                  \emph{Mathematics} (Masters)\\
  2011--present & UVSQ, \textbf{Lecturer} in \emph{Computer Science} (Bachelor,
                  Masters)\\
  2009--2010 & Université Paris Diderot,
               \textbf{Teaching assistant} (moniteur) in \emph{Computer Science}.\\
  2007--2009  & École Polytechnique, \textbf{Teaching assistant} in
                \emph{Computer Science}.
\end{tabular}

\subsubsection{Taught courses }

\begin{description}
  \setlength{\itemsep}{-0.5ex}
\item[Applications de l'informatique.]
  {\em Bachelor (L1) in Computer Science}.\\
  \url{https://defeo.lu/in202}.
\item[Algorithmique pour la cryptographie.]
  {\em Bachelor (L2) in Computer Science}.\\
  \url{https://defeo.lu/in420}.
\item[Mathématiques pour l’informatique.]
  {\em Bachelor (L2) in Computer Science}.\\
  \url{https://defeo.lu/in310}.
\item[Mathématiques.]
  {\em First year engineering school}.\\
  \url{https://defeo.lu/esb1math}.
\item[Applications web et sécurité.]
  {\em Masters (M1) in Computer Science} and {\em Second year engineering school}.\\
  \url{https://defeo.lu/aws}.
\item[Analyse d’algorithmes, programmation.]
  {\em Masters (M1) in Mathematics}.\\
  \url{https://defeo.lu/M1-AlgoProg/}.
\item[Algorithmique et programmation C]
  {\em Masters (M2) in Mathematics} (Algèbre appliquée).\\
  \url{https://defeo.lu/MA2-ace/}.
\item[Algèbre Commutative et effective]
  {\em Masters (M2) in Mathematics} (Algèbre appliquée).\\
  \url{https://defeo.lu/MA2-AlgoC/}.
\end{description}


\section*{Research administration}

\paragraph{Program committee member.}\

\begin{itemize}
\item
  ``AfricaCrypt 2019'',
\item
  \href{http://waifi.org/}{``International Workshop on the Arithmetic
    of Finite Fields (WAIFI) 2018''},
\item
  \href{http://issac-conference.org/2018/}{``International Symposium
    on Symbolic and Algebraic Computation (ISSAC) 2018''},
\item
  ``AfricaCrypt 2018''
\item
  \href{http://waifi.org/}{``International Workshop on the Arithmetic
    of Finite Fields (WAIFI) 2016''},
\item
  \href{http://yacc.univ-tln.fr/}{``Yet Another Conference on
    Cryptography (YACC) 2012''}.
\end{itemize}

\paragraph{Rewiever for various international journals and conferences.}

\paragraph{Research animation.}\

\begin{tabular}{l p{0.7\linewidth}}
  2017--2019 & Organisation of the
               \href{http://jncf.math.cnrs.fr/}{\textbf{Journées Nationales du Calcul
               Formel}}.\\
  2016 & Organisation of the
         \href{https://ecole-c2-2016.inria.fr/}{\textbf{C2 summer school}}.\\
  2014
             & Organisation of the workshop
               \href{https://idealcodes.github.io/clic-2014}{\textbf{CLIC 2014}} (Codes,
               Lattices, Ideals, Cryptography).\\
  2011--present & Université de
                  Versailles, organization of the
                  \textbf{Cryptology
                  seminar}.
\end{tabular}

\section*{Publication list}

\subsection*{Journal articles}
\begin{itemize}
\item
Ludovic Brieulle, Luca De~Feo, Javad Doliskani, Jean-Pierre Flori, and {\'E}ric
  Schost.
\newblock Computing isomorphisms and embeddings of finite fields.
\newblock {\em Mathematics of Computation}, 2018.
\newblock \url{https://doi.org/10.1090/mcom/3363}.

\item
Luca De~Feo, Cyril Hugounenq, J{\'e}r{\^o}me Pl{\^u}t, and {\'E}ric Schost.
\newblock Explicit isogenies in quadratic time in any characteristic.
\newblock {\em LMS Journal of Computation and Mathematics}, 19(A):267--282,
  2016.
  \newblock \url{http://dx.doi.org/10.1112/S146115701600036X}.
  
\item
Luca De~Feo, David Jao, and J{\'e}r{\^o}me Pl{\^u}t.
\newblock Towards quantum-resistant cryptosystems from supersingular elliptic
  curve isogenies.
\newblock {\em Journal of Mathematical Cryptology}, 8(3):209--247, 2014.
\newblock \url{http://dx.doi.org/10.1515/jmc-2012-0015}.

\item
Luca De~Feo and {\'E}ric Schost.
\newblock Fast arithmetics in {A}rtin-{S}chreier towers over finite fields.
\newblock {\em Journal of Symbolic Computation}, 47(7):771--792, 2012.
\newblock \url{http://dx.doi.org/10.1016/j.jsc.2011.12.008}.

\item
Luca De~Feo.
\newblock Fast algorithms for computing isogenies between ordinary elliptic
  curves in small characteristic.
\newblock {\em Journal of Number Theory}, 131(5):873--893, May 2011.
\newblock \url{http://dx.doi.org/10.1016/j.jnt.2010.07.003}.

\item
  Luca De~Feo, and Éric Schost.
  \newblock transalpyne: a language for automatic transposition.
  \newblock {\em ACM SIGSAM Bulletin}, 2010, 44 (1/2), pp. 59-71.
  \newblock \url{http://dx.doi.org/10.1145/1838599.1838624}.
\end{itemize}


\subsection*{In conference proceedings}
\begin{itemize}
\item
Luca De~Feo, Jean Kieffer, and Benjamin Smith.
\newblock Towards practical key exchange from ordinary isogeny graphs.
\newblock In {\em AsiaCrypt 2018}, 2018.
\newblock To appear, \url{https://eprint.iacr.org/2018/485}.

\item
  Luca De~Feo, Michael Kohlhase, Dennis Müller, Markus Pfeiffer, Florian Rabe, Nicolas M. Thiéry, Victor Vasilyev and Tom Wiesing.
  \newblock Knowledge-Based Interoperability for Mathematical Software Systems.
  \newblock {\em Mathematical Aspects of Computer and Information Sciences}, Springer, 2017, pp. 195-210.
  \newblock \url{http://dx.doi.org/10.1007/978-3-319-72453-9_14}.

\item
Luca De~Feo, Javad Doliskani, and \'{E}ric Schost.
\newblock Fast arithmetic for the algebraic closure of finite fields.
\newblock In {\em Proceedings of the 39th International Symposium on Symbolic
  and Algebraic Computation}, ISSAC '14, pages 122--129, New York, NY, USA,
  2014. ACM.
  \newblock \url{http://dx.doi.org/10.1145/2608628.2608672}.
  
\item
Luca De~Feo, Javad Doliskani, and Eric Schost.
\newblock Fast algorithms for $\ell$-adic towers over finite fields.
\newblock In {\em Proceedings of the 38th International Symposium on Symbolic
  and Algebraic Computation}, ISSAC '13, pages 165--172, New York, NY, USA,
  2013. ACM.
  \newblock \url{http://dx.doi.org/10.1145/2465506.2465956}.
  
\item
David Jao and Luca De~Feo.
\newblock Towards {Quantum-Resistant} cryptosystems from supersingular elliptic
  curve isogenies.
\newblock In Bo-Yin Yang, editor, {\em Post-Quantum Cryptography}, volume 7071
  of {\em Lecture Notes in Computer Science}, pages 19--34, Berlin, Heidelberg,
  2011. Springer Berlin / Heidelberg.
  \newblock \url{http://dx.doi.org/10.1007/978-3-642-25405-5_2}.
  
\item
Luca De~Feo and \'{E}ric Schost.
\newblock Fast arithmetics in {A}rtin-{S}chreier towers over finite fields.
\newblock In {\em ISSAC '09: Proceedings of the 2009 international symposium on
  Symbolic and algebraic computation}, pages 127--134, New York, NY, USA, 2009.
ACM.
  \newblock \url{}.
\end{itemize}

\subsection*{Preprints}
\begin{itemize}
\item
Luca De~Feo and Steven~D. Galbraith.
\newblock Seasign: Compact isogeny signatures from class group actions.
\newblock Cryptology ePrint Archive, Report 2018/824, 2018.
\newblock \url{https://eprint.iacr.org/2018/824}.

\item
  Luca De~Feo, Christophe Petit, and Michaël Quisquater.
  \newblock Applications of the affine geometry of $\mathrm{GF}(q^n)$ to root finding.
  \newblock In preparation. Best poster award ISSAC 2015.
  \newblock \url{https://github.com/defeo/root_finding/}.
\end{itemize}

\subsection*{Other}
\begin{itemize}

\item
Reza Azarderakhsh, Brian Koziel, Matt Campagna, Brian LaMacchia, Craig
  Costello, Patrick Longa, Luca De~Feo, Michael Naehrig, Basil Hess, Joost
  Renes, Amir Jalali, Vladimir Soukharev, David Jao, and David Urbanik.
  \newblock Supersingular isogeny key encapsulation.
  \newblock {\em Encryption/Key Encapsulation scheme, candidate to the NIST Post-Quantum Cryptography competion}, 2017.
  \newblock \url{https://sike.org}.

\item
  Luca De~Feo.
  \newblock Mathematics of Isogeny Based Cryptography.
  \newblock Lecture notes, {\em École Mathématique Africaine}, Thiès, Sénégal, 2017, 44 pp.
  \newblock \url{https://arxiv.org/abs/1711.04062}.
\end{itemize}

\subsection*{Thesis}
\begin{itemize}
\item Luca De~Feo.
\newblock Fast Algorithms for Towers of Finite Fields and Isogenies.
\newblock PhD Thesis. Advisors: François Morain, Éric Schost.
\newblock École Polytechnique, December 13, 2010. In English.
\newblock \url{https://hal.inria.fr/tel-00547034}.
\end{itemize}


\end{document}

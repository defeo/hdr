\documentclass{report}

\usepackage[b5paper]{geometry}
\usepackage[english]{babel}
\usepackage{array}
\usepackage{amsmath,amsthm,amsfonts,amssymb}
\usepackage{unicode}
\usepackage[type={CC},modifier={by-nc-sa},imagemodifier={-eu},version={4.0},imagewidth=5em]{doclicense}
\usepackage{fancyhdr}

\usepackage{algorithmic}
\renewcommand{\algorithmicrequire}{\textbf{Input:}}
\renewcommand{\algorithmicensure}{\textbf{Output:}}
\algsetup{linenodelimiter=.}

\usepackage[pdfusetitle]{hyperref}
\hypersetup{
  unicode=true,
  colorlinks=true,
  citecolor=blue!70!black,
  filecolor=black,
  linkcolor=red!70!black,
  urlcolor=blue,
  pdfstartview={FitH},
  pdfauthor={Luca De Feo},
  pdfsubject={Mathematics},
  pdfkeywords={Cryptography, Number theory, Computer algebra, Elliptic curves, Isogenies, Finite fields},
}

\usepackage{tikz}
\usetikzlibrary{arrows,matrix,decorations,decorations.text,decorations.pathmorphing,calc}
\pgfkeys{/triangle/.code=\tikzset{x={(-0.5cm,-0.866cm)},y={(1cm,0cm)}}}
\pgfkeys{/lattice/.code n args={4}{\tikzset{cm={#1,#2,#3,#4,(0,0)}}}}

% theorem environments
\theoremstyle{plain}
\newtheorem{theorem}{Theorem}
\newtheorem{lemma}[theorem]{Lemma}
\newtheorem{corollary}[theorem]{Corollary}
\newtheorem{proposition}[theorem]{Proposition}
\theoremstyle{definition}
\newtheorem{definition}[theorem]{Definition}
\newtheorem{example}[theorem]{Example}
\newtheorem{problem}{Problem}

\RequirePackage{amsmath,amsfonts}

\DeclareMathOperator{\End}{End}
\DeclareMathOperator{\Tr}{Tr}
\DeclareMathOperator{\Gal}{Gal}
\DeclareMathOperator{\ord}{ord}
\DeclareMathOperator{\loglog}{loglog}
\DeclareMathOperator{\GL}{GL}
\DeclareMathOperator{\SL}{SL}
\DeclareMathOperator{\Cl}{Cl}

\def\R{\ensuremath{\mathbb{R}}}
\def\A{\ensuremath{\mathbb{A}}}
\def\P{\ensuremath{\mathbb{P}}}
\def\F{\ensuremath{\mathbb{F}}}
\def\O{\ensuremath{\mathcal{O}}}
\def\tildO{\ensuremath{\tilde{O}}}
\def\euler{\ensuremath{\varphi}}


\title{Habilitation à diriger des recherches}
\author{Luca De Feo}

\begin{document}

\maketitle
\thispagestyle{fancy}
\renewcommand{\headrulewidth}{0pt}
\renewcommand{\footrulewidth}{0.4pt}
\cfoot{\doclicenseThis}
\lfoot{\LaTeX{} source code available at \url{https://github.com/defeo/hdr/}.}

%%%%%%%%%%%%%%%%%%%%%%%%%%%%%%%%%%%%%%%%%%%%%%%%%%%%%%%%%%%%%%%%

\chapter{Intro, etc.}

We'll take a journey, from one lonely elliptic curve over a finite
field to the amazingly rich world of isogeny graphs.

%%%%%%%%%%%%%%%%%%%%%%%%%%%%%%%%%%%%%%%%%%%%%%%%%%%%%%%%%%%%%%%%

\chapter{Finite fields and elliptic curves}

As short as possible, maybe write after, maybe an appendix.

- projective space, projective varieties,
- elliptic curves, torsion
- function fields, algebraic maps, frobenius, degree


%%%%%%%%%%%%%%%%%%%%%%%%%%%%%%%%%%%%%%%%%%%%%%%%%%%%%%%%%%%%%%%%

\chapter{The neighborhood}

% \begin{verse}
% You climb out the chimney\\
% And meet me in the middle\\
% The middle of the town\\
% And since there's no one else around,\\
% We let our hair grow long and forget all we used to know\\
% Then our skin gets thicker from living out in the snow
%
% Arcade Fire, The neighborhood #1 (Tunnels)
% \end{verse}

%%%%%%%%%%%%%%%%%%%%%%%%%%%%%%%%%%%%%%%%%%%%%%%%%%%%%%%%%%%%%%%%

\section{Isogenies}

An isogeny is a non-constant algebraic map between elliptic curves,
preserving the point at infinity. %
An isogeny is also a surjective group morphism of elliptic curves. %
It turns out these definitions are equivalent, but, before getting
these pages drenched in more properties and theorems, let's have a
look at an example.

The map $ϕ$ from the elliptic curve $y^2=x^3+x$ to $y^2=x^3-4x$
defined by
\begin{equation}
  \label{eq:isog-example}
  \begin{aligned}
    ϕ(x,y) &= \left(\frac{x^2+1}{x},y\frac{x^2-1}{x^2}\right),\\
    ϕ(\O) &= \O
  \end{aligned}
\end{equation}
is an isogeny. %
As an algebraic map it has degree $2$, which implies that it is a
two-to-one map, as it can be inferred from the polynomial degrees. %
Our rational maps are not defined at $x=0$, but the first expression
should really be understood as the projective map
\begin{equation*}
  ϕ(X:Y:Z) = \bigl(X(X^2+Z^2):Y(X^2-Z^2):ZX^2\bigr),
\end{equation*}
showing that the only other point in $\ker ϕ$, besides $\O$, is
$(0:0:1)$.


\begin{figure}
  \centering
  \begin{tikzpicture}[x=0.03\textwidth,y=0.03\textwidth]
    \begin{scope}
      \node[anchor=center] at (0,7) {$E \;:\; y^2 = x^3 + x$};

      \draw[thin,gray] (0,-6) -- (0,6);
      \draw[thin,gray] (-6,0) -- (6,0);

      \foreach \x/\y in {0/0,5/3,-4/3,-3/5,-2/1,-1/3} {
        \draw[blue,fill] (\x,\y) circle (0.2) node(E_\x_\y){}
        (\x,-\y) circle (0.2) node(E_\x_-\y){};
      }
    \end{scope}

    \draw[black!10!white,thick] (8,-7) -- +(0,14);
    
    \begin{scope}[shift={(16,0)}]
      \node at (0,7) {$E' \;:\; y^2 = x^3 - 4x$};

      \draw[thin,gray] (0,-6) -- (0,6);
      \draw[thin,gray] (-6,0) -- (6,0);

      \foreach \x/\y in {0/0,2/0,3/2,4/2,6/4,-2/0,-1/5} {
        \draw[color=blue,fill] (\x,\y) circle (0.2) node(F_\x_\y){}
        (\x,-\y) circle (0.2) node(F_\x_-\y){};
      }
    \end{scope}

    \begin{scope}[color=red,-latex,dashed]
        \path
        (E_5_3) edge (F_3_2)
        (E_-4_3) edge (F_4_-2)
        (E_-3_5) edge (F_4_2)
        (E_-2_1) edge (F_3_-2)
        (E_-1_3) edge (F_-2_0);
        \path
        (E_5_-3) edge (F_3_-2)
        (E_-4_-3) edge (F_4_2)
        (E_-3_-5) edge (F_4_-2)
        (E_-2_-1) edge (F_3_2)
        (E_-1_-3) edge (F_-2_0);
    \end{scope}
  \end{tikzpicture}
  \caption{The isogeny $(x,y) \mapsto \bigl((x^2+1)/x,\;y(x^2-1)/x^2\bigr)$,
    as a map between curves defined over $\F_{11}$.}
  \label{fig:isog-example}
\end{figure}


What does an isogeny ``look like''? %
Drawing the above one in $\R^2$ would look rather messy, but an
isogeny defined over the rationals is still an isogeny if we reduce
modulo a prime $p$. %
Figure~\ref{fig:isog-example} plots the action of the
isogeny~\eqref{eq:isog-example} on the image of the curves in
$\F_{11}$. %
A red arrow indicates that a point of the left curve is sent onto a
point on the right curve; the action on the point in $(0,0)$, going to
the point at infinity, is not shown. %
We observe a symmetry with respect to the $x$-axis, a consequence of
the fact that $ϕ$ is a group morphism; and, by looking closer, we may
also notice that collinear points are sent to collinear points, also a
necessity for a group morphism. %

Something strikes us, though: the map looks by no means surjective! %
This is because, when we think of isogenies, we think of them as
geometric objects, acting on the extension of the curves to the
algebraic closure. %
This is not dissimilar from the way power-by-$n$ maps act on the
multiplicative group $k^\times$ of a field $k$: the map $x↦x^2$, for
example, is a two-to-one (algebraic) group morphism on
$\F_{11}^\times$, and those elements that have no preimage, the
non-squares, will have exactly two square roots in $\F_{11^2}$, and so
on. %
In much the same way, in an algebraic closure $\bar{\F}_{11}$ of
$\F_{11}$, the isogeny $ϕ$ becomes surjective and every point gains
exactly two antecedents. %
This analogy is more profound that it may seem, and shall bear its
fruits in Chapter~\ref{cha:fpbar}.

For elliptic curves defined over a field of characteristic $p>0$,
there is another kind of isogeny. %
Let $E:y^2=x^3+ax+b$ be an elliptic curve, let $q$ be a power of $p$, and let
$E^{(q)}:y^2=x^3+a^qpx+b^q$. %
The isogeny $π_q:E\to E^{(q)}$ defined by
\begin{equation}
  \begin{aligned}
    π_q(x,y) &= (x^q,y^q),\\
    π_q(\O) &= \O
  \end{aligned}
\end{equation}
is a \emph{purely inseparable} isogeny of degree $q$. %
We call $π_q$ a \emph{Frobenius isogeny}. %
Despite being of degree $q$, Frobenius isogenies have trivial kernel,
and are one-to-one over finite fields (and other perfect fields). %

% Plotting the action of $π_p$ on the curve of
% Figure~\ref{fig:isog-example} would not be very telling, since in this
% case $E^{(p)}=E$ and the map acts like the identity on $\F_{11}$. %
% However $π_p$ is an important map, called the \emph{Frobenius
%   endomorphism} of $E$, and often denoted simply by $π$. %
% It permutes the points of $E/\bar{\F}_{11}$ in a non trivial way,
% reflecting the action of the Galois group of $\bar{\F}_{11}/\F_{11}$
% on $E$. %

Any isogeny can be decomposed as a product of a Frobenius isogeny and
a \emph{separable} isogeny:
\begin{equation*}
  \begin{tikzpicture}
    \node(E) at (0,0) {$E$};
    \node(Ep) at (2,0) {$E^{(q)}$};
    \node(E') at (4,0) {$E'$};
    \draw[->,auto] (E) edge node{\small $π_q$} (Ep)
    (Ep) edge node{\small $ϕ_s$} (E')
    (E) edge[bend right=20] node[below]{\small $ϕ$} (E');
  \end{tikzpicture}
\end{equation*}
Computing this decomposition is also easy given rational functions for
$ϕ$: simply factor out the powers of $p$ from the polynomials. %
For these reasons we shall be mostly concerned with separable
isogenies and their computations.

The most unique property of separable isogenies is that they are 
entirely determined by their kernel. %

\begin{proposition}
  Let $E$ be an elliptic curve, and let $G$ be a finite subgroup of
  $E$. %
  There are a unique elliptic curve $E'$, and a unique separable
  isogeny $ϕ$, such that $\ker ϕ=G$ and $ϕ:E\to E'$. %
\end{proposition}

Said otherwise, for any finite subgroup $G⊂E$, we have an exact
sequence of algebraic groups
\begin{equation*}
  0 \to G \to E \overset{ϕ}{\to} E' \to 0,
\end{equation*}
which justifies the notation $E/G$ for the image curve $E'$. %
Conversely, since any non-constant morphism of algebraic curves
necessarily has finite kernel, we have a canonical bijection between
the finite subgroups of a curve $E$ and the isogenies with domain
$E$. %
This correspondence is rich in consequences: it is an easy exercise to
prove the following useful facts. %

\begin{corollary}\ 
  \begin{enumerate}
  \item Any isogeny can be decomposed as a product of prime degree
    isogenies.
  \item Let $E$ be defined over an algebraically closed field $k$, let
    $ℓ$ be a prime different from the characteristic of $k$, then
    there are exactly $ℓ+1$ isogenies of degree $ℓ$ with domain $E$.
  \end{enumerate}
\end{corollary}

Slightly less more work is required to prove the following,
fundamental, theorem (the difficulty comes essentially from the
inseparable part, see~\cite[III.6.1]{silverman:elliptic} for a
detailed proof).

\begin{theorem}[Dual isogeny theorem]
  Let $ϕ:E\to E'$ be an isogeny of degree $m$. %
  There is a unique isogeny $\hat{ϕ}:E'\to E$ such that
  \[\hat{ϕ}∘ϕ = [m]_E, \quad ϕ∘\hat{ϕ} = [m]_{E'}.\] %
  $\hat{ϕ}$ is called the \emph{dual isogeny of $ϕ$}; it has the
  following properties:
  
  \begin{enumerate}
  \item $\hat{ϕ}$ is defined over $k$ if and only if $ϕ$ is;
  \item $\widehat{ψ∘ϕ} = \hat{ϕ}∘\hat{ψ}$ for any isogeny $ψ:E'\to E''$;
  \item $\widehat{ψ+ϕ} = \hat{ψ} + \hat{ϕ}$ for any isogeny $ψ:E\to E'$;
  \item $\deg ϕ = \deg\hat{ϕ}$;
  \item $\hat{\hat{ϕ}} = ϕ$.
  \end{enumerate}
\end{theorem}


The computational counterpart to the kernel-isogeny correspondence, is
given by V\'elu's much celebrated formulas. %

\begin{proposition}[{V\'elu~\cite{velu71}}]
  \label{th:velu}
  Let $E:y^2=x^3+ax+b$ be an elliptic curve defined over a field $k$,
  and let $G⊂E(\bar{k})$ be a finite subgroup. %
  The separable isogeny $ϕ:E\to E/G$, of kernel $G$, can be written as
  \begin{equation*}
    ϕ(P) = \left(
      x(P) + \sum_{Q∈G\setminus\{\O\}}x(P+Q)-x(Q),
      y(P) + \sum_{Q∈G\setminus\{\O\}}y(P+Q)-y(Q)
    \right);
  \end{equation*} %
  and the curve $E/G$ has equation $y^2=x^3+a'x+b'$, where
  \begin{align*}
    a' &= a - 5\sum_{Q∈G\setminus\{\O\}}(3x(Q)^2+a),\\
    b' &= b - 7\sum_{Q∈G\setminus\{\O\}}(5x(Q)^3+3ax(Q)+b).
  \end{align*}
\end{proposition}

In summary, knowledge of its kernel allows us to compute an isogeny,
and \emph{vice versa}. %
This will be crucial when we will study isogeny-based cryptosystems
in Chapter~\ref{cha:crypto}.

%%%%%%%%%%%%%%%%%%%%%%%%%%%%%%%%%%%%%%%%%%%%%%%%%%%%%%%%%%%%%%%%

\section{The explicit isogeny and other problems}

When it comes to computations, V\'elu's formulas are only part of the
story. %
Instead of being given a kernel $G$ and wanting to compute the curve
$E/G$, we may ask the opposite question. %
Elkies, while working on point counting~\cite{elkies92,elkies98},
famously baptized this the \emph{explicitly isogeny problem}.

\begin{problem}[Explicit isogeny problem]
  Let $E$ and $E'$ be two elliptic curves, and $ℓ$ an integer. %
  Decide whether there exists an isogeny $ϕ:E\to E'$ of degree $ℓ$,
  and compute its kernel.
\end{problem}

- How to compute an isogeny,
- Modular polynomials, application to point counting

\section{Curling up}

- Isogeny graphs, Serre trees?
- Isogeny volcanoes, complex multiplication
- Supersingular graphs

%%%%%%%%%%%%%%%%%%%%%%%%%%%%%%%%%%%%%%%%%%%%%%%%%%%%%%%%%%%%%%%%
%%%%%%%%%%%%%%%%%%%%%%%%%%%%%%%%%%%%%%%%%%%%%%%%%%%%%%%%%%%%%%%%
%%%%%%%%%%%%%%%%%%%%%%%%%%%%%%%%%%%%%%%%%%%%%%%%%%%%%%%%%%%%%%%%

\chapter{Panoptycon}
\label{cha:fpbar}

- towers of finite fields,
- isomorphisms, embeddings,
- Fp-bar
- lattices

%%%%%%%%%%%%%%%%%%%%%%%%%%%%%%%%%%%%%%%%%%%%%%%%%%%%%%%%%%%%%%%%
%%%%%%%%%%%%%%%%%%%%%%%%%%%%%%%%%%%%%%%%%%%%%%%%%%%%%%%%%%%%%%%%
%%%%%%%%%%%%%%%%%%%%%%%%%%%%%%%%%%%%%%%%%%%%%%%%%%%%%%%%%%%%%%%%

\chapter{Crypto and quantum computers}
\label{cha:crypto}

- Expander graphs
- HHS, CSIDH
- SIDH
- Signatures?
- Quantum algorithms? Kuperberg?

%%%%%%%%%%%%%%%%%%%%%%%%%%%%%%%%%%%%%%%%%%%%%%%%%%%%%%%%%%%%%%%%
%%%%%%%%%%%%%%%%%%%%%%%%%%%%%%%%%%%%%%%%%%%%%%%%%%%%%%%%%%%%%%%%
%%%%%%%%%%%%%%%%%%%%%%%%%%%%%%%%%%%%%%%%%%%%%%%%%%%%%%%%%%%%%%%%

\clearpage
\bibliographystyle{plain}
\bibliography{hdr}

\end{document}

% LocalWords:  isogeny isogenies morphism surjective projective
% LocalWords:  preimage bijection cryptosystems

\documentclass{report}

\usepackage[b5paper]{geometry}
\usepackage[english]{babel}
\usepackage{array}
\usepackage{amsmath,amsthm,amsfonts,amssymb}
\usepackage{unicode}
\usepackage[type={CC},modifier={by-nc-sa},imagemodifier={-eu},version={4.0},imagewidth=5em]{doclicense}
\usepackage{fancyhdr}

\usepackage{algorithmic}
\renewcommand{\algorithmicrequire}{\textbf{Input:}}
\renewcommand{\algorithmicensure}{\textbf{Output:}}
\algsetup{linenodelimiter=.}

\usepackage[pdfusetitle]{hyperref}
\hypersetup{
  unicode=true,
  colorlinks=true,
  citecolor=blue!70!black,
  filecolor=black,
  linkcolor=red!70!black,
  urlcolor=blue,
  pdfstartview={FitH},
  pdfauthor={Luca De Feo},
  pdfsubject={Mathematics},
  pdfkeywords={Cryptography, Number theory, Computer algebra, Elliptic curves, Isogenies, Finite fields},
}

\usepackage{tikz}
\usetikzlibrary{arrows,matrix,decorations,decorations.text,decorations.pathmorphing,calc}
\pgfkeys{/triangle/.code=\tikzset{x={(-0.5cm,-0.866cm)},y={(1cm,0cm)}}}
\pgfkeys{/lattice/.code n args={4}{\tikzset{cm={#1,#2,#3,#4,(0,0)}}}}

% theorem environments
\theoremstyle{plain}
\newtheorem{theorem}{Theorem}
\newtheorem{lemma}[theorem]{Lemma}
\newtheorem{corollary}[theorem]{Corollary}
\newtheorem{proposition}[theorem]{Proposition}
\theoremstyle{definition}
\newtheorem{definition}[theorem]{Definition}
\newtheorem{example}[theorem]{Example}
\newtheorem{problem}{Problem}

\RequirePackage{amsmath,amsfonts}

\DeclareMathOperator{\End}{End}
\DeclareMathOperator{\Tr}{Tr}
\DeclareMathOperator{\Gal}{Gal}
\DeclareMathOperator{\ord}{ord}
\DeclareMathOperator{\loglog}{loglog}
\DeclareMathOperator{\GL}{GL}
\DeclareMathOperator{\SL}{SL}
\DeclareMathOperator{\Cl}{Cl}

\def\R{\ensuremath{\mathbb{R}}}
\def\A{\ensuremath{\mathbb{A}}}
\def\P{\ensuremath{\mathbb{P}}}
\def\F{\ensuremath{\mathbb{F}}}
\def\O{\ensuremath{\mathcal{O}}}
\def\tildO{\ensuremath{\tilde{O}}}
\def\euler{\ensuremath{\varphi}}

\newcommand{\leg}[2]{\left(\frac{#1}{#2}\right)}

\title{Promenades in isogeny graphs}
%\subtitle{Habilitation à diriger des recherches}
\author{Luca De Feo}

\begin{document}

\maketitle
\thispagestyle{fancy}
\renewcommand{\headrulewidth}{0pt}
\renewcommand{\footrulewidth}{0.4pt}
\cfoot{\doclicenseThis}
\lfoot{\LaTeX{} source code available at \url{https://github.com/defeo/hdr/}.}

%%%%%%%%%%%%%%%%%%%%%%%%%%%%%%%%%%%%%%%%%%%%%%%%%%%%%%%%%%%%%%%%

\chapter{Intro, etc.}

We'll take a journey, from one lonely elliptic curve over a finite
field to the amazingly rich world of isogeny graphs.

%%%%%%%%%%%%%%%%%%%%%%%%%%%%%%%%%%%%%%%%%%%%%%%%%%%%%%%%%%%%%%%%

\chapter{Finite fields and elliptic curves}

As short as possible, maybe write after, maybe an appendix.

- projective space, projective varieties,
- elliptic curves, torsion
- function fields, algebraic maps, frobenius, degree
- differentials? pullbacks?
- modular polynomial?

%%%%%%%%%%%%%%%%%%%%%%%%%%%%%%%%%%%%%%%%%%%%%%%%%%%%%%%%%%%%%%%%

\chapter{The neighborhood}

% \begin{verse}
% You climb out the chimney\\
% And meet me in the middle\\
% The middle of the town\\
% And since there's no one else around,\\
% We let our hair grow long and forget all we used to know\\
% Then our skin gets thicker from living out in the snow
%
% Arcade Fire, The neighborhood #1 (Tunnels)
% \end{verse}

%%%%%%%%%%%%%%%%%%%%%%%%%%%%%%%%%%%%%%%%%%%%%%%%%%%%%%%%%%%%%%%%

\section{Isogenies}

An isogeny is a non-constant algebraic map between elliptic curves,
preserving the point at infinity. %
An isogeny is also a surjective group morphism of elliptic curves. %
It turns out these definitions are equivalent, but, before getting
these pages drenched in more properties and theorems, let's have a
look at an example.

The map $ϕ$ from the elliptic curve $y^2=x^3+x$ to $y^2=x^3-4x$
defined by
\begin{equation}
  \label{eq:isog-example}
  \begin{aligned}
    ϕ(x,y) &= \left(\frac{x^2+1}{x},y\frac{x^2-1}{x^2}\right),\\
    ϕ(\O) &= \O
  \end{aligned}
\end{equation}
is an isogeny. %
As an algebraic map it has degree $2$, which implies that it is a
two-to-one map, as it can be inferred from the polynomial degrees. %
Our rational maps are not defined at $x=0$, but the first expression
should really be understood as the projective map
\begin{equation*}
  ϕ(X:Y:Z) = \bigl(X(X^2+Z^2):Y(X^2-Z^2):ZX^2\bigr),
\end{equation*}
showing that the only other point in $\ker ϕ$, besides $\O$, is
$(0:0:1)$.


\begin{figure}
  \centering
  \begin{tikzpicture}[x=0.03\textwidth,y=0.03\textwidth]
    \begin{scope}
      \node[anchor=center] at (0,7) {$E \;:\; y^2 = x^3 + x$};

      \draw[thin,gray] (0,-6) -- (0,6);
      \draw[thin,gray] (-6,0) -- (6,0);

      \foreach \x/\y in {0/0,5/3,-4/3,-3/5,-2/1,-1/3} {
        \draw[blue,fill] (\x,\y) circle (0.2) node(E_\x_\y){}
        (\x,-\y) circle (0.2) node(E_\x_-\y){};
      }
    \end{scope}

    \draw[black!10!white,thick] (8,-7) -- +(0,14);
    
    \begin{scope}[shift={(16,0)}]
      \node at (0,7) {$E' \;:\; y^2 = x^3 - 4x$};

      \draw[thin,gray] (0,-6) -- (0,6);
      \draw[thin,gray] (-6,0) -- (6,0);

      \foreach \x/\y in {0/0,2/0,3/2,4/2,6/4,-2/0,-1/5} {
        \draw[color=blue,fill] (\x,\y) circle (0.2) node(F_\x_\y){}
        (\x,-\y) circle (0.2) node(F_\x_-\y){};
      }
    \end{scope}

    \begin{scope}[color=red,-latex,dashed]
        \path
        (E_5_3) edge (F_3_2)
        (E_-4_3) edge (F_4_-2)
        (E_-3_5) edge (F_4_2)
        (E_-2_1) edge (F_3_-2)
        (E_-1_3) edge (F_-2_0);
        \path
        (E_5_-3) edge (F_3_-2)
        (E_-4_-3) edge (F_4_2)
        (E_-3_-5) edge (F_4_-2)
        (E_-2_-1) edge (F_3_2)
        (E_-1_-3) edge (F_-2_0);
    \end{scope}
  \end{tikzpicture}
  \caption{The isogeny $(x,y) \mapsto \bigl((x^2+1)/x,\;y(x^2-1)/x^2\bigr)$,
    as a map between curves defined over $\F_{11}$.}
  \label{fig:isog-example}
\end{figure}


What does an isogeny ``look like''? %
Drawing the above one in $\R^2$ would look rather messy, but an
isogeny defined over the rationals is still an isogeny if we reduce
modulo a prime $p$. %
Figure~\ref{fig:isog-example} plots the action of the
isogeny~\eqref{eq:isog-example} on the image of the curves in
$\F_{11}$. %
A red arrow indicates that a point of the left curve is sent onto a
point on the right curve; the action on the point in $(0,0)$, going to
the point at infinity, is not shown. %
We observe a symmetry with respect to the $x$-axis, a consequence of
the fact that $ϕ$ is a group morphism; and, by looking closer, we may
also notice that collinear points are sent to collinear points, also a
necessity for a group morphism. %

Something strikes us, though: the map looks by no means surjective! %
This is because, when we think of isogenies, we think of them as
geometric objects, acting on the extension of the curves to the
algebraic closure. %
This is not dissimilar from the way power-by-$n$ maps act on the
multiplicative group $k^\times$ of a field $k$: the map $x↦x^2$, for
example, is a two-to-one (algebraic) group morphism on
$\F_{11}^\times$, and those elements that have no preimage, the
non-squares, will have exactly two square roots in $\F_{11^2}$, and so
on. %
In much the same way, in an algebraic closure $\bar{\F}_{11}$ of
$\F_{11}$, the isogeny $ϕ$ becomes surjective and every point gains
exactly two antecedents. %
This analogy is more profound that it may seem, and shall bear its
fruits in Chapter~\ref{cha:fpbar}.

For elliptic curves defined over a field of characteristic $p>0$,
there is another kind of isogeny. %
Let $E:y^2=x^3+ax+b$ be an elliptic curve, let $q$ be a power of $p$, and let
$E^{(q)}:y^2=x^3+a^qpx+b^q$. %
The isogeny $π_q:E\to E^{(q)}$ defined by
\begin{equation}
  \begin{aligned}
    π_q(x,y) &= (x^q,y^q),\\
    π_q(\O) &= \O
  \end{aligned}
\end{equation}
is a \emph{purely inseparable} isogeny of degree $q$. %
We call $π_q$ a \emph{Frobenius isogeny}. %
Despite being of degree $q$, Frobenius isogenies have trivial kernel,
and are one-to-one over finite fields (and other perfect fields). %

% Plotting the action of $π_p$ on the curve of
% Figure~\ref{fig:isog-example} would not be very telling, since in this
% case $E^{(p)}=E$ and the map acts like the identity on $\F_{11}$. %
% However $π_p$ is an important map, called the \emph{Frobenius
%   endomorphism} of $E$, and often denoted simply by $π$. %
% It permutes the points of $E/\bar{\F}_{11}$ in a non trivial way,
% reflecting the action of the Galois group of $\bar{\F}_{11}/\F_{11}$
% on $E$. %

Any isogeny can be decomposed as a product of a Frobenius isogeny and
a \emph{separable} isogeny:
\begin{equation*}
  \begin{tikzpicture}
    \node(E) at (0,0) {$E$};
    \node(Ep) at (2,0) {$E^{(q)}$};
    \node(E') at (4,0) {$E'$};
    \draw[->,auto] (E) edge node{\small $π_q$} (Ep)
    (Ep) edge node{\small $ϕ_s$} (E')
    (E) edge[bend right=20] node[below]{\small $ϕ$} (E');
  \end{tikzpicture}
\end{equation*}
Computing this decomposition is also easy given rational functions for
$ϕ$: simply factor out the powers of $p$ from the polynomials. %
For these reasons we shall be mostly concerned with separable
isogenies and their computations.

The most unique property of separable isogenies is that they are 
entirely determined by their kernel. %

\begin{proposition}
  Let $E$ be an elliptic curve, and let $G$ be a finite subgroup of
  $E$. %
  There is a curve $E'$, and a separable isogeny $ϕ$, such that
  $\ker ϕ=G$ and $ϕ:E\to E'$. %
  Furthermore, $E'$ and $ϕ$ are unique up to composition with an
  isomorphism $E'≃E''$. %
\end{proposition}

Said otherwise, for any finite subgroup $G⊂E$, we have an exact
sequence of algebraic groups
\begin{equation*}
  0 \to G \to E \overset{ϕ}{\to} E' \to 0.
\end{equation*}
Uniqueness up to isomorphisms justifies the notation $E/G$ for the
isomorphism class of the image curve $E'$. %
Now, it would be useful if we could find a way to define a canonical
representative inside $E/G$. %
It turns out there is a pretty natural way to define one.

\begin{definition}[Normalized isogeny]
  Let $E,E'$ be two elliptic curves, $ω_E,ω_E'$ their \emph{invariant
    differential}, $ϕ:E\to E'$ a separable isogeny and
  $ϕ^*:Ω_{E'}\to Ω_E$ its \emph{pullback}. %
  We say that $ϕ$ is \emph{normalized} if its pullback preserves the
  invariant differentials, i.e., $ϕ^*(ω_{E'})=ω_E$. %
\end{definition}

Since $ϕ$ is separable, $ϕ^*$ is an isomorphism of vector spaces of
dimension one. %
I.e., if $ϕ$ is not normalized, then it is only ``off'' by a
(non-zero) constant $ϕ^*(ω_{E'})=cω_E$, and we can easily normalize
$ϕ$ by a change of variables. %
This also shows that, for fixed $E$ and $\ker ϕ$, the normalized
isogeny is unique, and justifies abusing the notation $E/G$ to mean
the image of the normalized isogeny with kernel $G$.\footnote{Note
  that this convention is not universal in the literature, as there
  are other useful choices for a canonical representative of $E/G$.}

Conversely, since any non-constant morphism of algebraic curves
necessarily has finite kernel, we have a canonical bijection between
the finite subgroups of a curve $E$ and the normalized isogenies with
domain $E$. %
This correspondence is rich in consequences: it is an easy exercise to
prove the following useful facts. %

\begin{corollary}\ 
  \begin{enumerate}
  \item Any isogeny can be decomposed as a product of prime degree
    isogenies.
  \item Let $E$ be defined over an algebraically closed field $k$, let
    $ℓ$ be a prime different from the characteristic of $k$, then
    there are exactly $ℓ+1$ normalized isogenies of degree $ℓ$ with
    domain $E$.
  \end{enumerate}
\end{corollary}

Slightly less more work is required to prove the following,
fundamental, theorem (the difficulty comes essentially from the
inseparable part, see~\cite[III.6.1]{silverman:elliptic} for a
detailed proof).

\begin{theorem}[Dual isogeny theorem]
  Let $ϕ:E\to E'$ be an isogeny of degree $m$. %
  There is a unique isogeny $\hat{ϕ}:E'\to E$ such that
  \[\hat{ϕ}∘ϕ = [m]_E, \quad ϕ∘\hat{ϕ} = [m]_{E'}.\] %
  $\hat{ϕ}$ is called the \emph{dual isogeny of $ϕ$}; it has the
  following properties:
  
  \begin{enumerate}
  \item $\hat{ϕ}$ has degree $m$;
  \item $\hat{ϕ}$ is defined over $k$ if and only if $ϕ$ is;
  \item $\widehat{ψ∘ϕ} = \hat{ϕ}∘\hat{ψ}$ for any isogeny $ψ:E'\to E''$;
  \item $\widehat{ψ+ϕ} = \hat{ψ} + \hat{ϕ}$ for any isogeny $ψ:E\to E'$;
  \item $\deg ϕ = \deg\hat{ϕ}$;
  \item $\hat{\hat{ϕ}} = ϕ$.
  \end{enumerate}
\end{theorem}

Note that, since $[m]^*(ω)=mω$, if $ϕ$ is normalized $\hat{ϕ}$ almost
never is. %
The computational counterpart to the kernel-isogeny correspondence, is
given by Vélu's much celebrated formulas. %

\begin{proposition}[{Vélu~\cite{velu71}}]
  \label{th:velu}
  Let $E:y^2=x^3+ax+b$ be an elliptic curve defined over a field $k$,
  and let $G⊂E(\bar{k})$ be a finite subgroup. %
  The normalized separable isogeny $ϕ:E\to E/G$, of kernel $G$, can be
  written as
  \begin{equation*}
    ϕ(P) = \left(
      x(P) + \sum_{Q∈G\setminus\{\O\}}x(P+Q)-x(Q),\\
      y(P) + \sum_{Q∈G\setminus\{\O\}}y(P+Q)-y(Q)
    \right);
  \end{equation*} %
  and the curve $E/G$ has equation $y^2=x^3+a'x+b'$, where
  \begin{align*}
    a' &= a - 5\sum_{Q∈G\setminus\{\O\}}(3x(Q)^2+a),\\
    b' &= b - 7\sum_{Q∈G\setminus\{\O\}}(5x(Q)^3+3ax(Q)+b).
  \end{align*}
\end{proposition}

But how ``good'' are Vélu's formulas from a computational
perspective? %
``Pretty good'', is the message we want to convey, but, in order to
understand the question, we need to discuss \emph{rationality}. %
Let $E$ be defined over a field $k$, with algebraic closure
$\bar{k}$. %
We say that an isogeny $ϕ:E\to E'$ is \emph{defined over $k$}, or
\emph{$k$-rational}, if $ϕ$ is invariant under the action of the
Galois group of $\bar{k}/k$. %
This is equivalent to $ϕ$ being defined by rational maps with
coefficients in $k$, and also to $\ker ϕ$ being stable under
$\Gal(\bar{k}/k)$. %
It implies that $E'$ is defined over $k$, but the converse is not
true. %

In the rest of this work, when we say ``an isogeny'', we really mean
``an isogeny defined over the base field'', unless specified
otherwise. %
What are the input and output sizes of Vélu's formulas, if we
restrict to $k$-rational isogenies? %

The output is a pair of rational fractions, and, letting $ℓ=\#G$, it
is not too difficult to see that they have $O(ℓ)$ coefficients. %
The input is the kernel $G$, and, since it is finite, it must be
generated by at most two elements. %
However $G$ is only stable under $\Gal(\bar{k}/k)$, implying that its
elements are defined in an (Abelian) extension of degree dividing
$ℓ$. %
Thus, in general, $G$ is represented by $\O(\ell)$ coefficients of $k$. %

Now, if we apply mindlessly Vélu's formulas, we need at list $O(ℓ^2)$
coefficients in $k$ to write down all the elements of $G$. %
A better approach is to represent $G$ by a polynomial with
coefficients in $k$ that vanishes on all $P∈G$, for example
\begin{equation*}
  h_G(x) = \prod_{P∈G\setminus\{\O\}} (x - x(P)).
\end{equation*}
The polynomial $h_G$ is usually called the \emph{kernel polynomial} of
$G$, and has coefficients in $k$ if and only if $ϕ$ is $k$-rational;
it can be computed from the generators of $G$ in only $\tildO(ℓ)$
operations over $k$. %
Following Kohel~\cite{kohel}, we can rewrite Vélu's formulas in terms
of $h_G$, then, their evaluation can also be accomplished in
$\tildO(ℓ)$ operations.

In conclusion, we see that Vélu's formula make the kernel/isogeny
correspondence explicit, using a quasi-optimal number of operations in
general. %
This will be crucial when we will study isogeny-based cryptosystems in
Chapter~\ref{cha:crypto}, however, we will encounter there some
examples where the cost of evaluating an isogeny is exponentially
lower than that of Vélu's formulas. %


%%%%%%%%%%%%%%%%%%%%%%%%%%%%%%%%%%%%%%%%%%%%%%%%%%%%%%%%%%%%%%%%

\section{The explicit isogeny and other problems}

When it comes to computations, Vélu's formulas are only part of the
story: how do we find a rational kernel $G$ in the first place?
Elkies, while working on point counting~\cite{elkies92,elkies98},
famously baptized this the \emph{explicit isogeny problem}.

\begin{problem}[Explicit isogeny problem]
  \label{prob:expl-isog}
  Let $E$ be an elliptic curve, and let $ℓ$ be an integer. %
  Find, if they exist, one or all isogenies of degree $ℓ$ with domain
  $E$.
\end{problem}

However a slightly modified version of the same problem can be often
found in the literature.

\begin{problem}
  \label{prob:expl-isog-2}
  Let $E$ and $E'$ be two elliptic curves, and $ℓ$ an integer. %
  Decide whether there exists an isogeny $ϕ:E\to E'$ of degree $ℓ$,
  and compute its kernel.
\end{problem}

The many variants of the explicit isogeny problem have kept the
research community busy for more than twenty years, and still do
today. %
Let's have a closer look at it. %

\subsection{Elkies' algorithm}

For a start, it should be noticed that both variants are by no means
``hard''. %
Indeed, we have explicit formulas for adding points on a curve $E$,
from which we can deduce an explicit formula for multiplying points on
$E$ by any scalar $ℓ∈ℤ$. %
Said otherwise, we have an explicitly formula for the
multiplication-by-$ℓ$ isogeny, and, by reading its denominators, we
can deduce a polynomial $ψ_ℓ(x,y)$ that vanishes precisely on $E[ℓ]$,
the \emph{$ℓ$-th division polynomial} of $E$
(see~\cite[III.4]{blake+seroussi+smart} for explicit formulas). %

Let's assume for simplicity that $ℓ$ is prime and different from the
characteristic, then we know there are at most $ℓ+1$ normalized
isogenies of degree $ℓ$ from $E$. %
Factoring $ψ_ℓ$ over the field of definition $k$ of $E$ lets us
compute all possible kernel polynomials of order $ℓ$, and thus all
possible isogenies. %
At most $ℓ+1$ applications of Vélu's formula will then give the answer
to either of the two variants of the explicit isogeny problem. %
$ψ_ℓ$ is none else than the kernel polynomial of the
multiplication-by-$ℓ$ isogeny, and, since $E[ℓ]$ contains $ℓ^2$
points, its degree is bounded by $ℓ^2$, thus this algorithm costs no
more than factoring a degree $ℓ^2$ polynomial in $k[x]$. %

We see that the matter is not solving the explicit isogeny
problem. The matter is solving it fast!

An interesting case, and the one Elkies was originally interested in,
is when the curves are defined over a finite field. %
Let's relax the problem a bit, and see what can be told about it. %
Decisional versions, first: two elliptic curves are said to be
\emph{isogenous} if there exists an isogeny connecting them (this is
an equivalence relation, thanks to the dual isogeny theorem).

\begin{problem}
  \label{prob:isogenous}
  Let $E,E'$ be two elliptic curves defined over a finite field
  $\F_q$, decide whether they are isogenous.
\end{problem}

Tate~\cite[Th.~1(c)]{Tate} famously showed\footnote{Tate, citing
  Mumford, also points out that, for the case of elliptic curves, this
  is an easy consequence of the much celebrated work of
  Deuring~\cite{deuring41}.} that $E$ and $E'$ are isogenous over
$\F_q$ if and only if $\#E(\F_q)=\#E'(\F_q)$. %
Schoof's point counting algorithm~\cite{schoof85,schoof95} completely
settles the problem by computing the orders of $E$ and $E'$ in
polynomial time in $\log(q)$. %
However, when we add a degree constraint, the problem immediately
becomes harder, even for finite fields. %

\begin{problem}
  \label{prob:ell-isogenous}
  Let $E,E'$ be two elliptic curves, and $ℓ$ be an integer. Decide
  whether they are $ℓ$-isogenous.
\end{problem}

The \emph{modular polynomial} helps solve this problem. %
Assuming $ℓ$ is prime, the $ℓ$-th modular polynomial, denoted by
$Φ_ℓ(X,Y)$, it is a bivariate polynomial with coefficients in $ℤ$,
symmetric in $X$ and $Y$, of degree $ℓ+1$ in each variable, with the
following property: two elliptic curves $E,E'$ are $ℓ$-isogenous if
and only if $Φ_ℓ(j(E),j(E'))=Φ_ℓ(j(E'),j(E))=0$. %
We stress that this is true over \textbf{any} field, and that the
definition of $Φ_ℓ$ is independent of the base field. %
Given that $Φ_ℓ$ has $O(ℓ^2)$ coefficients (and rather large ones),
using it to decide the explicit isogeny problem is asymptotically only
slightly better than factoring the division polynomial; it is however
usually considerably more efficient in practice, especially when
tables of modular polynomials are precomputed, as is the case in
computer algebra systems such as Pari~\cite{Pari}, Magma~\cite{MAGMA},
or SageMath~\cite{Sage}. %

The modular polynomial can also be used to produce all isogenous
elliptic curves, up to isomorphism, to a given curve: simply plug
$j(E)$ in $Φ_ℓ$, then factor $Φ_ℓ(j(E),Y)$ to find the isogenous
$j$-invariants. %
Elkies used this approach to reduce the explicit isogeny problem to
Problem~\ref{prob:expl-isog-2}, but he managed to extract even more
information from $Φ_ℓ$: he showed how to obtain a \emph{normalized
  equation} for the image curve.

\begin{theorem}[{\cite{elkies92,schoof95,elkies98}}]
  Let $E:y^2=x^3+ax+b$ be an elliptic curve, let $j$ be its
  $j$-invariant and let $j'$ be such that $Φ_ℓ(j,j')=0$. %
  Assume that $(∂Φ_ℓ/∂Y)(j,j')≠0$, and define
  \begin{equation}
    \label{eq:elkies-modpol}
    \begin{aligned}
      λ &= \frac{-18}{ℓ}\frac{b}{a}\frac{\frac{∂Φ_ℓ}{∂X}(j,j')}{\frac{∂Φ_ℓ}{∂Y}(j,j')}j,\\
      a' &= -\frac{1}{48}\frac{λ^2}{j'(j'-1728)}\frac{1}{ℓ^4},\\
      b' &= -\frac{1}{864}\frac{λ^3}{(j')^2(j'-1728)}\frac{1}{ℓ^6}.
    \end{aligned}
  \end{equation}
  Then there is a normalized isogeny of degree $ℓ$ from $E$ to
  $E':y^2=x^3+a'x+b'$.
\end{theorem}

Elkies' theorem prompts us to define a weaker variant of the explicit
isogeny problem.

\begin{problem}[Inverse Vélu problem\footnote{The name is ours and
    not attested in the literature.}]
  Let $E,E'$ be elliptic curves such that there exists a normalized
  isogeny $ϕ:E\to E'$ of degree $ℓ$. %
  Compute the kernel of $ϕ$.
\end{problem}

Unsurprisingly, Elkies also gave the solution to this problem
in~\cite{elkies92,elkies98}. %
He observed that the rational fractions defining $ϕ$ are related by a
differential equation, involving only the coefficients of $E$ and
$E'$. %
Solving the differential equation gives the rational fractions, and
thus the kernel. %
This gives a method to solve the inverse Vélu problem in $O(ℓ^2)$
operations over the base field, or even $\tildO(ℓ)$ using the
optimizations of Bostan, Morain, Salvy and
Schost~\cite{bostan+morain+salvy+schost08}. %

We have, essentially, sketched the computation involved in the
Schoof-Elkies-Atkin point counting algorithm~\cite{schoof95}, for
those that are called \emph{Elkies primes}. %
However, the last part of Elkies' algorithm, the solution to the
inverse Vélu problem, only works when the characteristic is $0$ or
\emph{large enough}. %
While this is good enough for counting points of elliptic curves
defined over a prime field $\F_p$, it fails, for example, over binary
fields. %

\subsection{Couveignes' algorithm}


The first 

- Couveignes, Lercier 
- Schoof


\section{Curling up}

- Isogeny graphs, Serre trees?
- Isogeny volcanoes, complex multiplication
- Supersingular graphs

%%%%%%%%%%%%%%%%%%%%%%%%%%%%%%%%%%%%%%%%%%%%%%%%%%%%%%%%%%%%%%%%
%%%%%%%%%%%%%%%%%%%%%%%%%%%%%%%%%%%%%%%%%%%%%%%%%%%%%%%%%%%%%%%%
%%%%%%%%%%%%%%%%%%%%%%%%%%%%%%%%%%%%%%%%%%%%%%%%%%%%%%%%%%%%%%%%

\chapter{Panoptycon}
\label{cha:fpbar}

- towers of finite fields,
- isomorphisms, embeddings,
- Fp-bar
- lattices

%%%%%%%%%%%%%%%%%%%%%%%%%%%%%%%%%%%%%%%%%%%%%%%%%%%%%%%%%%%%%%%%
%%%%%%%%%%%%%%%%%%%%%%%%%%%%%%%%%%%%%%%%%%%%%%%%%%%%%%%%%%%%%%%%
%%%%%%%%%%%%%%%%%%%%%%%%%%%%%%%%%%%%%%%%%%%%%%%%%%%%%%%%%%%%%%%%

\chapter{Crypto and quantum computers}
\label{cha:crypto}

- Expander graphs
- HHS, CSIDH
- SIDH
- Signatures?
- Quantum algorithms? Kuperberg?

%%%%%%%%%%%%%%%%%%%%%%%%%%%%%%%%%%%%%%%%%%%%%%%%%%%%%%%%%%%%%%%%
%%%%%%%%%%%%%%%%%%%%%%%%%%%%%%%%%%%%%%%%%%%%%%%%%%%%%%%%%%%%%%%%
%%%%%%%%%%%%%%%%%%%%%%%%%%%%%%%%%%%%%%%%%%%%%%%%%%%%%%%%%%%%%%%%

\clearpage
\bibliographystyle{plain}
\bibliography{hdr}

\end{document}

% LocalWords:  isogeny isogenies morphism surjective projective
% LocalWords:  preimage bijection cryptosystems univariate Decisional
% LocalWords:  isomorphisms isogenous Schoof's decisional bivariate
% LocalWords:  precomputed  Abelian Vélu's

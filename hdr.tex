\documentclass{report}

\usepackage[b5paper]{geometry}
\usepackage[english]{babel}
\usepackage{array}
\usepackage{amsmath,amsthm,amsfonts,amssymb}
\usepackage{unicode}
\usepackage[type={CC},modifier={by-nc-sa},imagemodifier={-eu},version={4.0},imagewidth=5em]{doclicense}
\usepackage{fancyhdr}

\usepackage{algorithmic}
\renewcommand{\algorithmicrequire}{\textbf{Input:}}
\renewcommand{\algorithmicensure}{\textbf{Output:}}
\algsetup{linenodelimiter=.}

\usepackage[pdfusetitle]{hyperref}
\hypersetup{
  unicode=true,
  colorlinks=true,
  citecolor=blue!70!black,
  filecolor=black,
  linkcolor=red!70!black,
  urlcolor=blue,
  pdfstartview={FitH},
  pdfauthor={Luca De Feo},
  pdfsubject={Mathematics},
  pdfkeywords={Cryptography, Number theory, Computer algebra, Elliptic curves, Isogenies, Finite fields},
}

\usepackage{tikz}
\usetikzlibrary{arrows,matrix,decorations,decorations.text,decorations.pathmorphing,calc}
\pgfkeys{/triangle/.code=\tikzset{x={(-0.5cm,-0.866cm)},y={(1cm,0cm)}}}
\pgfkeys{/lattice/.code n args={4}{\tikzset{cm={#1,#2,#3,#4,(0,0)}}}}

\RequirePackage{amsmath,amsfonts}

\DeclareMathOperator{\End}{End}
\DeclareMathOperator{\Tr}{Tr}
\DeclareMathOperator{\Gal}{Gal}
\DeclareMathOperator{\ord}{ord}
\DeclareMathOperator{\loglog}{loglog}
\DeclareMathOperator{\GL}{GL}
\DeclareMathOperator{\SL}{SL}
\DeclareMathOperator{\Cl}{Cl}

\def\R{\ensuremath{\mathbb{R}}}
\def\A{\ensuremath{\mathbb{A}}}
\def\P{\ensuremath{\mathbb{P}}}
\def\F{\ensuremath{\mathbb{F}}}
\def\O{\ensuremath{\mathcal{O}}}
\def\tildO{\ensuremath{\tilde{O}}}
\def\euler{\ensuremath{\varphi}}

\newcommand{\leg}[2]{\left(\frac{#1}{#2}\right)}

\title{Habilitation à diriger des recherches}
\author{Luca De Feo}

\begin{document}

\maketitle
\thispagestyle{fancy}
\renewcommand{\headrulewidth}{0pt}
\renewcommand{\footrulewidth}{0.4pt}
\cfoot{\doclicenseThis}
\lfoot{\LaTeX{} source code available at \url{https://github.com/defeo/hdr/}.}

%%%%%%%%%%%%%%%%%%%%%%%%%%%%%%%%%%%%%%%%%%%%%%%%%%%%%%%%%%%%%%%%

\chapter{Intro, etc.}

We'll take a journey, from one lonely elliptic curve over a finite
field to the amazingly rich world of isogeny graphs.

%%%%%%%%%%%%%%%%%%%%%%%%%%%%%%%%%%%%%%%%%%%%%%%%%%%%%%%%%%%%%%%%

\chapter{Finite fields and elliptic curves}

As short as possible, maybe write after, maybe an appendix.

- projective space, projective varieties,
- elliptic curves
- function fields, algebraic maps


%%%%%%%%%%%%%%%%%%%%%%%%%%%%%%%%%%%%%%%%%%%%%%%%%%%%%%%%%%%%%%%%

\chapter{The neighborhood}

% \begin{verse}
% You climb out the chimney\\
% And meet me in the middle\\
% The middle of the town\\
% And since there's no one else around,\\
% We let our hair grow long and forget all we used to know\\
% Then our skin gets thicker from living out in the snow
%
% Arcade Fire, The neighborhood #1 (Tunnels)
% \end{verse}



\section{The explicit isogeny problem}

An isogeny is a non-constant algebraic map between elliptic curves,
preserving the point at infinity. %
An isogeny is also a surjective group morphism of elliptic curves. %
It turns out these definitions are equivalent, but, before getting
these pages drenched in more properties and theorems, let's have a
look at an example.

The map $φ$ from the elliptic curve $y^2=x^3+x$ to $y^2=x^3-4x$
defined by
\begin{equation}
  \label{eq:isog-example}
  \begin{aligned}
    φ(x,y) &= \left(\frac{x^2+1}{x},y\frac{x^2-1}{x^2}\right),\\
    φ(\O) &= \O
  \end{aligned}
\end{equation}
is an isogeny. %
As an algebraic map it has degree $2$, which implies that it is a
two-to-one map, as it can be inferred from the polynomial degrees. %
Our rational maps are not defined at $x=0$, but this expression should
really be understood as the projective map
\begin{equation*}
  φ(X:Y:Z) = (X(X^2+Z^2):Y(X^2-Z^2):ZX^2),
\end{equation*}
showing that the only other point in $\ker φ$, besides $\O$, is
$(0:0:1)$.


\begin{figure}
  \centering
  \begin{tikzpicture}[x=0.03\textwidth,y=0.03\textwidth]
    \begin{scope}
      \node[anchor=center] at (0,7) {$E \;:\; y^2 = x^3 + x$};

      \draw[thin,gray] (0,-6) -- (0,6);
      \draw[thin,gray] (-6,0) -- (6,0);

      \foreach \x/\y in {0/0,5/3,-4/3,-3/5,-2/1,-1/3} {
        \draw[blue,fill] (\x,\y) circle (0.2) node(E_\x_\y){}
        (\x,-\y) circle (0.2) node(E_\x_-\y){};
      }
    \end{scope}

    \draw[black!10!white,thick] (8,-7) -- +(0,14);
    
    \begin{scope}[shift={(16,0)}]
      \node at (0,7) {$E' \;:\; y^2 = x^3 - 4x$};

      \draw[thin,gray] (0,-6) -- (0,6);
      \draw[thin,gray] (-6,0) -- (6,0);

      \foreach \x/\y in {0/0,2/0,3/2,4/2,6/4,-2/0,-1/5} {
        \draw[color=blue,fill] (\x,\y) circle (0.2) node(F_\x_\y){}
        (\x,-\y) circle (0.2) node(F_\x_-\y){};
      }
    \end{scope}

    \begin{scope}[color=red,-latex,dashed]
        \path
        (E_5_3) edge (F_3_2)
        (E_-4_3) edge (F_4_-2)
        (E_-3_5) edge (F_4_2)
        (E_-2_1) edge (F_3_-2)
        (E_-1_3) edge (F_-2_0);
        \path
        (E_5_-3) edge (F_3_-2)
        (E_-4_-3) edge (F_4_2)
        (E_-3_-5) edge (F_4_-2)
        (E_-2_-1) edge (F_3_2)
        (E_-1_-3) edge (F_-2_0);
    \end{scope}
  \end{tikzpicture}
  \caption{The isogeny $(x,y) \mapsto \bigl((x^2+1)/x,\;y(x^2-1)/x^2\bigr)$,
    as a map between curves defined over $\F_{11}$.}
  \label{fig:isog-example}
\end{figure}


What does an isogeny ``look like''? %
Drawing the above one in $\R^2$ would look rather messy, but an
isogeny defined over the rationals is still an isogeny if we reduce
modulo a prime $p$. %
Figure~\ref{fig:isog-example} plots the action of the
isogeny~\eqref{eq:isog-example} on the image of the curves in
$\F_{11}$. %
We observe a symmetry with respect to the $x$-axis, a consequence of
the fact that $φ$ is a group morphism; and by looking closer we may
also notice that colinear points are sent to colinear points, also a
necessity for a group morphism. %
Something strikes us, though: the map looks by no means surjective! %
This is because, when we think of isogenies, we think of them as
geometric objects, acting on the extension of the curves to the
algebraic closure. %
This is not dissimilar from the way power-by-$n$ maps act on the
multiplicative group $k^\times$ of a field $k$: the map $x↦x^2$, for
example, is a two-to-one map on $\F_{11}$, and those elements that
have no antecedents, the non-squares, will have exactly two in
$\F_{11^2}$, and so on. %
In much the same way, in an algebraic closure $\bar{\F}_{11}$ of
$\F_{11}$, the isogeny $φ$ becomes surjective and every point gains
exactly two antecedents. %
This analogy is more profound that it may seem, and shall bear its
fruits in Chapter~\ref{cha:fpbar}.

- Isogeny definitions


- How to compute an isogeny,
- Modular polynomials, application to point counting

\section{Curling up}

- Isogeny graphs, Serre trees?
- Isogeny volcanoes, complex multiplication
- Supersingular graphs

%%%%%%%%%%%%%%%%%%%%%%%%%%%%%%%%%%%%%%%%%%%%%%%%%%%%%%%%%%%%%%%%

\chapter{Panoptycon}
\label{cha:fpbar}

- towers of finite fields,
- isomorphisms, embeddings,
- Fp-bar
- lattices

%%%%%%%%%%%%%%%%%%%%%%%%%%%%%%%%%%%%%%%%%%%%%%%%%%%%%%%%%%%%%%%%

\chapter{Crypto and quantum computers}

- Expander graphs
- HHS, CSIDH
- SIDH
- Signatures?
- Quantum algorithms? Kuperberg?

%%%%%%%%%%%%%%%%%%%%%%%%%%%%%%%%%%%%%%%%%%%%%%%%%%%%%% 
\clearpage
\bibliographystyle{plain}
\bibliography{hdr}

\end{document}

% LocalWords:  isogeny isogenies morphism surjective projective
% LocalWords:  colinear preimage
